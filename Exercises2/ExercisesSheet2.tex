\documentclass[12pt,letterpaper]{article}
\usepackage{./preamble}

%%%%%%%%%%%%%%%%%%%%%%%%%%%%%%%%%%%%%%%%%%
%%%% Edit These for yourself
%%%%%%%%%%%%%%%%%%%%%%%%%%%%%%%%%%%%%%%%%%
\newcommand\course{Computational Statistics}
\newcommand\hwnumber{2}
\newcommand\userID{Davi Sales Barreira}
\DeclareRobustCommand{\rchi}{{\mathpalette\irchi\relax}}
\newcommand{\irchi}[2]{\raisebox{\depth}{$#1\chi$}}

\begin{document}
% \textbf{\Large Worksheet completed with Octave.}

\section*{Exercise 1 (Monte Carlo for Gaussians)}
\begin{enumerate}[leftmargin=!,labelindent=5pt]
	\item Let's prove that $E[\phi(X)] = E[\phi(X+\theta)
	exp(\frac{-1}{2}\theta^T\theta - \theta^T X)]$.
	% $$ E[\phi(X)] = \int_{\mathbb{R}^d} \phi(x) \pi(x) dx_1...dx_d$$

	$$ E[\phi(X+\theta)exp(\frac{-1}{2}\theta^T\theta - \theta^T X)]
	= \int_{\mathbb{R}^d} \phi(x+\theta) exp(\frac{-1}{2}\theta^T
	\theta - \theta^T X)\pi(x)dx_1...dx_d = $$

	$$ \propto \int_{\mathbb{R}^d} \phi(x+\theta) exp\left(\frac{-1}{2}
	\theta^T
	\theta - \theta^T X \right)exp(-x^T x / 2)dx_1...dx_d = $$

	$$ \int_{\mathbb{R}^d} \phi(x+\theta) exp\left(\frac{-1}{2}
	(x-\theta)^T(x-\theta)\right)
	dx_1...dx_d$$

	Finally, making $x-\theta = y$,
	$$ \int_{\mathbb{R}^d} \phi(y) exp\left(\frac{-1}{2}
	(y)^T(y)\right)
	dx_1...dx_d = E[\phi(Y)] $$

\end{enumerate}

\end{document}
