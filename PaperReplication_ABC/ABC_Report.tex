% This is samplepaper.tex, a sample chapter demonstrating the
% LLNCS macro package for Springer Computer Science proceedings;
% Version 2.20 of 2017/10/04
%
\documentclass[runningheads]{llncs}
%
\usepackage{float}
\usepackage{natbib}
\usepackage{amsfonts}
\usepackage{amsmath, bm}
\usepackage{booktabs} % For pretty tables
\usepackage{caption} % For caption spacing
\usepackage{subcaption} % For sub-figures
\usepackage{graphicx}
\usepackage{pgfplots}
\usepackage[all]{nowidow}
\usepackage[utf8]{inputenc}
\usepackage{tikz}
\usetikzlibrary{er,positioning,bayesnet}
\usepackage{multicol}
% \usepackage{algpseudocode,algorithm,algorithmicx}
\usepackage[ruled,vlined]{algorithm2e}
\usepackage{algpseudocode,algorithmicx}
% \usepackage{minted}
\usepackage{hyperref}
\usepackage[inline]{enumitem} % Horizontal lists
% Used for displaying a sample figure. If possible, figure files should
% be included in EPS format.
%
% If you use the hyperref package, please uncomment the following line
% to display URLs in blue roman font according to Springer's eBook style:
% \renewcommand\UrlFont{\color{blue}\rmfamily}

\newcommand{\card}[1]{\left\vert{#1}\right\vert}
\newcommand*\Let[2]{\State #1 $\gets$ #2}
\definecolor{blue}{HTML}{1F77B4}
\definecolor{orange}{HTML}{FF7F0E}
\definecolor{green}{HTML}{2CA02C}

\pgfplotsset{compat=1.14}

\renewcommand{\topfraction}{0.85}
\renewcommand{\bottomfraction}{0.85}
\renewcommand{\textfraction}{0.15}
\renewcommand{\floatpagefraction}{0.8}
\renewcommand{\textfraction}{0.1}
\setlength{\floatsep}{3pt plus 1pt minus 1pt}
\setlength{\textfloatsep}{3pt plus 1pt minus 1pt}
\setlength{\intextsep}{3pt plus 1pt minus 1pt}
\setlength{\abovecaptionskip}{2pt plus 1pt minus 1pt}

\begin{document}
%
\title{Approximate Bayesian Computation Overview}
%
%\titlerunning{Abbreviated paper title}
% If the paper title is too long for the running head, you can set
% an abbreviated paper title here
%
\author{Davi Sales Barreira}
%
%\authorrunning{F. Author et al.}
% First names are abbreviated in the running head.
% If there are more than two authors, 'et al.' is used.
%
\institute{FGV - Escola de Matemática Aplicada, Rio de Janeiro, Brasil\\ 
\email{davisbarreira@gmail.com}}
%
\maketitle              % typeset the header of the contribution
%
\begin{abstract}
Approximate Bayesian Computation (ABC)
methods are known as likelihood-free techniques, thus are a useful
approach in problems that the likelihood is intractable, e.g., likelihood
not available in closed form, or likelihood too expensive to calculate.
In this article, we present an overview of the method by replicating
the paper Approximate Bayesian computational
methods by \cite{Marin2012}.

\keywords{Approximate Bayesian Computation \and likelihood-free \and
computational statistics.}
\end{abstract}
%
%
%
\section{Introduction}
\subsection{Original ABC} \label{subsec:statistical-summaries}

The Approximate Bayesian Computation method was originally described
by \citet{Rubin1984} as a thought experiment to explain how to sample
from a posterior distribution with a frequency interpretation.
The method became proeminent due to the fact that it circumvents
the need to calculate the likelihood function in order to
obtain the posterior distribution. This can be a very useful
feature in scenarios where the likelihood is intractable or
too expensive to calculate. One example is in the case where
one has latent variables, thus, the likelihood is expressed as:

\begin{equation}
  \ell(\bm\theta \mid \bm y) =
  \bm\int \ell^*(\bm\theta \mid \bm y, \bm u) d\bm u
\end{equation}
with $\bm y$ being the observed variable,
$\bm u$ the latent variable and $\bm\theta$ is the parameter of interest.

\citet{Tavare505} introduced the ABC algorithm as a rejection
techinique to obtain the posterior distribution without the explicit
calculation of the likelihood. This original algorithm is given below.

\hfill\break
\begin{algorithm}[H]
\SetAlgoLined
\For{i=1 to N}{
 \Repeat{$\bm y = \bm z$}{
    Sample $\bm\theta' \sim \pi(\cdot)$

    Generate $\bm z \sim p(\cdot \mid \bm\theta')$
 }
 Set $\bm \theta_i = \bm \theta'$
  
}
 \caption{Original ABC method}
\end{algorithm}
The proof that the algorithm indeed results in an iid sample
from the posterior is shown below. Let $\bm y$ be the observed,
$\bm \theta$ the parameter of interest and $\bm z$ the generated
samples.

\begin{equation}
  f(\bm \theta_i) \propto \sum_{\bm z \in \mathbb{D}}
  \pi(\bm \theta_i) p(\bm z \mid \bm \theta_i) \mathbb I_{\bm y}(\bm z)
  = \pi(\bm \theta_i) p(\bm y \mid \bm \theta_i) \propto
  \pi(\bm \theta_i \mid \bm y)
\end{equation}

The original ABC formulation only works for
the case where $\bm y$ is discrete
taking finite values, and therefore, an exact match is possible
to be obtained in a finite number of simulations. \citet{Pritchard1999}
then extended the method to a more general form considering an
approximation instead of an exact match. This extended algorithm is
shown below, where

\begin{itemize}
  \item[--] $\eta$ is a function defining a statistic (e.g. the mean),
  \item[--] $\rho$ is a distance function,
  \item[--] $\epsilon$ is an acceptance tolerance.
\end{itemize}

\hfill\break
\begin{algorithm}[H]
\SetAlgoLined
\For{i=1 to N}{
 \Repeat{$\rho[\eta(\bm y) , \eta (\bm z)] \leq \epsilon$}{
    Sample $\bm\theta' \sim \pi(\cdot)$

    Generate $\bm z \sim p(\cdot \mid \bm\theta')$
 }
 Set $\bm \theta_i = \bm \theta'$
  
}
 \caption{ABC method for discrete and continuous distributions}
\end{algorithm}
\hfill\break

For this ABC algorithm, instead of the actual posterior,
we get

\begin{equation}
\pi_\epsilon(\bm \theta, \bm z \mid \bm y) = 
\frac{\pi(\bm \theta) p(\bm z \mid \bm \theta)
\mathbb I_{A_{\epsilon,\bm y}}(\bm z)}
{\int_{A_{\epsilon,\bm y}\times \bm\theta}\pi(\bm \theta)
p(\bm z \mid \bm \theta)d\bm z d \bm \theta}
\end{equation}
where, $A_{\epsilon,\bm y} = \{
\bm z \in \mathcal D \mid \rho[\eta(\bm z), \eta(\bm y) \leq \epsilon]
\}$.
Hence, for a tolerance ($\epsilon$) ``small enough", we expect a good
approximation of the real posterior.
\begin{equation}
\pi_\epsilon(\bm \theta \mid \bm y) = 
\int \pi_\epsilon(\bm \theta, \bm z \mid \bm y) d \bm z \approx
\pi(\bm \theta \mid \bm y)
\end{equation}


\subsection{Moving Average} \label{subsec:statistical-summaries}

We will use the Moving Average model, also denoted as MA(q),
for assessing the performance of the ABC methods. The MA(q) process
is a stochastic process defined by:

\begin{equation}
y_k = u_k + \sum_{i=1}^q \theta_i u_{k-i}
\end{equation}
where $(u_k)_{k \in \mathbb Z} \overset{iid}{\sim} N(0,1)$.
The true posterior distribution of MA(2) and MA(1) models can be
numerically computed, since the likelihood function is indeed avaliable.
Therefore, the approximations obtained through ABC can be compared with
the true posterior. The marginal posterior distributions are also
obtained numerically.

For $q=2$, imposing the standard identifiability condition
we obtain the following conditions:

\begin{equation}
-2 < \theta_1 < 2, \quad \quad \theta_1+\theta_2 > -1, \quad \quad
\theta_1 - \theta_2 < 1.
\end{equation}
hence, we use an uniform distribution over this triangular region as
prior for $\bm \theta$.

We generate a synthetic sample of length 100 using
$(\theta_1, \theta_2) = (0.6, 0.2)$. For $q=2$, the 
true posterior has the following form:
\begin{equation}
\pi(\bm\theta \mid \bm y) \propto \pi(\bm\theta)
p(\bm y \mid \bm \theta), \quad \quad
\bm y \mid \bm \theta \sim MVN(0, \Sigma) \quad
\end{equation}

$$
\Sigma =
\left[ 
\begin{smallmatrix}
 1+\theta_1^2 + \theta_2^2    & \theta_1 + \theta_2 \theta_1 & \theta_2                     & 0                            & 0        & 0 & ... & 0 \\
 \theta_1 + \theta_2 \theta_1 & 1+\theta_1^2 + \theta_2^2    & \theta_1 + \theta_2 \theta_1 & \theta_2                     & 0        & 0 & ... & 0 \\
 \theta_2                     & \theta_1 + \theta_2 \theta_1 & 1+\theta_1^2 + \theta_2^2    & \theta_1 + \theta_2 \theta_1 & \theta_2 & 0 & ... & 0 \\
 0               & \theta_2   & \theta_1 + \theta_2 \theta_1 & 1+\theta_1^2 + \theta_2^2    & \theta_1 + \theta_2 \theta_1 & \theta_2 &... & 0 \\
 \vdots & \vdots & \vdots & \vdots & \vdots & \vdots & \vdots & \vdots \\
 0 & 0 & 0 & 0 & 0 & \theta_2 & \theta_1 + \theta_1\theta_2 & 1+\theta_1^2 + \theta_2^2 \\
\end{smallmatrix}
\right]
$$

For this model, applying the ABC algorithm consisted in the
following steps:
\begin{itemize}
  \item Sample $\bm \theta ^ *$ from the uniform triangular prior
  using rejection sampling;
  \item For each $k \in \{-1,0, 1, ..., 100 \}$, sample
  $u_k \overset{iid}{\sim} N(0,1)$.
  \item For each $k \in \{ 1, 2, ..., 100\}$, calculate 
  $z_k = u_k + \sum^2_{i=1}\theta_i^* u_{k-i}$.
\end{itemize}

Two distance metrics were initially compared.
The raw distance between the series

\begin{equation}
\rho^2\{ \bm z, \bm y\} = \sum^{n=100}_{k=1}(y_k - z_k)^2
\end{equation}
and the sum of the quadratic distances between the first $q = 2$
autocovariances.
\begin{equation}
\tau_j(\bm x) = \sum^{n=100}_{k = j+1} x_k x_{k-j}, \quad \quad
\rho^2 = \sum^{q=2}_{j=0}(\tau_j(\bm y) - \tau_j(\bm z))^2
\end{equation}

Below we present the results of running ABC for the MA(2) process
using the autocovariances distance.

  \begin{figure}[H]
      \centering
      \includegraphics[width=10cm]{images/ABCmodel1.png}
      \caption{Comparison between the true posterior
      (\textit{line in black}), with the samples produced using the ABC
      . The number of simulations is $N = 10^6$,
      and the threshold $\epsilon$ corresponds to the quantile of
      accepting 0.1\%. The $\rho$ used was the distance
      of the autocovariances.}
  \end{figure}

\section{ABC Calibration}
\subsection{Summary Statistics ($\eta$)}
\label{subsec:statistical-summaries}

As the number of observations
grow, using the raw distance between each observation becomes too
prohibitive, due to the rarity of actually
obtaining samples close to each observation.
The alternative is to try using summary statistics of low dimension.
The ideal case is using sufficient statistics, which guarantees that the
method indeed approximates the true posterior.
The problem is that
low-dimensional sufficient statistics are rarely available. Hence,
choosing an appropriate low-dimensional statistic is paramount for
obtaining good approximations with ABC \citep{Marin2012}.

\citet{Beaumont2018} separates the approaches to address this
problem into two categories: one is optimally choosing
subsets of summary statistcs, and the other is projecting
a set of summary statistics onto lower dimensional maps.

In the first category, \citet{Joyce2008} introduced the concept
of approximate sufficiency. The main idea is that given a set
of summary statistics $s \subset S$, an approximately sufficient subset can
be found by sequantially including those statistics into the ABC
target. The method develops a score written as
\begin{equation}
\delta_k = sup_\theta \{
\log f(s_k \mid s_1,...,s_{k-1},\theta)
\}
- inf_\theta\{
\log f(s_k \mid s_1,...,s_{k-1},\theta)
\}
\end{equation}
and tests whether $\delta_k$ is less than a given tolerance. In the
case this is true, the statistic is deemed approximately sufficient.

\citet{Marin2012} criticize this method.
They state that the construction of the statistics is not
discussed in the paper by \citet{Joyce2008}. Secondly, the
order in which the statistics are tested may alter the final subset.
And finally, that the corrections proposed do not address the impact
of correlation between the summary statistics.

In the second category,
\citet{fearnhead2010constructing} propose a way of constructing
appropriate summary statistics for ABC in a semiautomatic manner.
Their method aims at minimizing the expected posterior loss
\begin{equation}
\mathbb E[(\bm\theta - \bm{\hat\theta})\mid \bm y]
\implies
\bm {\hat\theta} = \mathbb E[\bm\theta \mid \bm y]
\end{equation}
hence, the optimal summary statistic is
\begin{equation}
s = \mathbb E[\bm \theta \mid \bm y]
\end{equation}
Since $E[\bm \theta \mid \bm y]$ is unknown, it can instead
be estimated by performing a linear regression on each
component of $\bm \theta$. Therefore, the single optimal
summary statistic is written as
\begin{equation}
s_{opt} =  \bm\beta^T \bm f(s)
\end{equation}
where $\bm\beta$ is the vector of the regression coefficients and
$\bm f(s)$ is the vector of summary statistics functions.


\subsection{Tolerance threshold($\epsilon$)}
\label{subsec:statistical-summaries}
The choice of $\epsilon$ is mostly driven by computational
limitations. The lower the value of $\epsilon$, the higher
the number of simulations required. The standard practice
\citep{Beaumont2012} is to chose $\epsilon$ as a quantile
of the simulated distance $\rho$, e.g., for $10^6$ simulations,
taking $\epsilon = 0.1\%$ corresponds to accepting
$10^3$ sampled $\bm \theta$'s. This implies that the choice
of $\epsilon$ is just a proxy for the number of simulations
to be performed.

\subsection{Calibration comparison in MA(2)}
\label{subsec:statistical-summaries}

Using the MA(2) model, we run the ABC algorithm comparing
different calibrations. As stated before, two different
summary statistics are used, the raw distance and the
autocovariances distance. Figure ~\ref{fig:calibration1}
makes it clear that
the autocovariances distance perfom better that the raw distance,
therefore, through the rest of this article we will only be using
the autocovariances distance.

Figure ~\ref{fig:calibration2} shows the improvement of the ABC
approximation with the decrease of the tolerance comparing the
marginal distributions of each parameter. Regarding $\theta_1$,
the method seems to be converging to the real distribution,
but for $\theta_2$ the approximation doesn't seem to be improving
much.


    \begin{figure}[H]
        \centering
        \includegraphics[width=11cm]{images/ABCmodel1_Comparison.png}
        \caption{Comparison of ABC method when using autocovariance
        distance 
        (\textit{left}) versus raw distance (\textit{right}).
        The number of simulations is $N = 10^6$ and different
        thresholds $\epsilon$ are used.}
        \label{fig:calibration1}
    \end{figure}

    \begin{figure}[H]
        \centering
        \includegraphics[width=11cm]{images/ABCmodel1_Marginal.png}
        \caption{Comparison of ABC samples with the true posterior
        marginal distribution for $\theta_1$ (\textit{left}) and
        $\theta_2$ (\textit{right}).
        }
        \label{fig:calibration2}
    \end{figure}


\newpage
\section{ABC Variations}
Over the years, the ``vanilla'' ABC algorithm has
been modified in order to get both better approximations
for the posterior
and lessen the need of using prohibitively small
threshold values, thus decreasing the number
of simulations necessary. In this section,
some of these variations are presented.
\subsection{MCMC-ABC}
\label{subsec:statistical-summaries}

Using non-informative priors is usually very inefficient,
because it leads to lots of rejections. To tackle this
problem, \citet{Marjoram2013} came up with MCMC-ABC.
The algorithm is presented below.

\hfill\break
\begin{algorithm}[H]
\SetAlgoLined
Use Algorithm 2 to get $(\bm \theta^{(0)}, \bm z^{(0)})$ from the
target $\pi_\epsilon(\bm \theta, \bm z \mid \bm y)$.

\For{i=1 to N}{
 \Repeat{$\rho[\eta(\bm y) , \eta (\bm z)] \leq \epsilon$}{
    Sample $\bm\theta'$ from the Markov kernel
    $q(\cdot \mid \theta^{(i-1)})$

    Generate $\bm z \sim p(\cdot \mid \bm\theta')$

    Sample $u \sim U[0,1]$

    \If{
      $u \leq \frac{
                    \pi(\bm\theta')q(\bm\theta^{(i-1)}\mid)}
          {\pi(\bm\theta^{(i-1)})q(\bm\theta^{(i-1)}\mid)}$ and
      $\rho\{ \eta(\bm z'),\eta(\bm y) \} \leq \epsilon$
    }
    {Set $(\bm \theta^{(i)}, \bm z^{(i)}) =
    (\bm \theta', \bm z')$}
    \Else
    {Set $(\bm \theta^{(i)}, \bm z^{(i)}) =
    (\bm \theta^{(i-1)}, \bm z^{(i-1)})$}
}
}
 \caption{MCMC-ABC}
\end{algorithm}
\hfill\break

As can be seen, the MCMC-ABC maintains the likelihood-free
feature of the original ABC method, but it also estimates
$\pi_\epsilon(\bm \theta \mid \bm y)$ instead of the true
posterior distribution. The initialisation of the algorithm
actually uses the ``vanilla" ABC method, thus the burn-in
of the first iterations can be avoided.

The MCMC-ABC method performs a bit better
for our MA(2) example, as shown
in Figure ~\ref{fig:mcmcabc}.

\newpage
\hfill\break
    \begin{figure}[H]
        \centering
        \includegraphics[width=9cm]{images/ABC-MCMC.png}
        \caption{Comparison of ABC-MCMC samples with the true posterior
        marginal distribution for $\theta_1$ (\textit{left}) and
        $\theta_2$ (\textit{right}) using $\epsilon = 0.1\%$.
        }
        \label{fig:mcmcabc}
    \end{figure}

\subsection{Noisy ABC}
\label{subsec:statistical-summaries}

Another variation of ABC is called Noisy ABC, that was
proposed by \citet{Wilkinson2013}. The original ABC algorithm can
be thought as a rejection algorithm using a uniform kernel
($\mathbb I_{A_{\epsilon, \bm y}(\bm z)}$). The \textit{Noisy} version
generalizes this, allowing the use of different kernels, hence:
\begin{equation}
\pi_\epsilon(\bm \theta, \bm z \mid \bm y) = 
\frac{\pi(\bm \theta) p(\bm z \mid \bm \theta)
K_\epsilon(\bm y - \bm z)}
{\int \pi(\bm \theta)
p(\bm z \mid \bm \theta)K_\epsilon(\bm y - \bm z)d\bm z d \bm \theta}
\end{equation}

Now, instead of accepting if
$\rho\{\eta(\bm y),\eta(\bm z)\} \leq \epsilon$, we accept with 
probability
$\frac{K_\epsilon(\bm y - \bm z)}{max\{K_\epsilon(\bm y - \bm z\})}
$.
\hfill\break

    \begin{figure}[H]
        \centering
        \includegraphics[width=9cm]{images/NoisyABC.png}
        \caption{
        Illustration of \textit{Noisy} ABC rejection kernels, where
        $s$ is the statistic from the ABC sampler and $s_{obs}$ is 
        the observed value from the data.
        Figure from \cite{Beaumont2018}.
        \label{fig:noisyabc}
        }
    \end{figure}

\subsection{Sequantial Techiniques}
\label{subsec:statistical-summaries}

Sequential techniques are also used with ABC to enhance the
efficiency of the algorithms. A popular method in this regard is
the ABC-PMC
(ABC population Monte Carlo) by \citet{Beaumont2009}.
It estimates the scale of the random walk step from the previous
simulations and
uses a sequence of tolerance thresholds
($\epsilon_1 \geq ... \geq \epsilon_T$) to approximate the distribution.

 A recent work by \citet{Simola2019} proposed a method
 for adaptively selecting this sequence of tolerances in a way that
 improves the computationl efficiency and defines a stopping rule,
 thus assisting in automating the termination of the sampling
 procedure. The algorithm is presented below.


\hfill\break

  \begin{algorithm}[H]
  \SetAlgoLined
  At iteration t=1,

  \For{i=1 to N}{
    \Repeat{$\rho[\eta(\bm y) , \eta (\bm z)] \leq \epsilon$}{
      Sample $\bm\theta_i^{(1)} \sim \pi(\cdot)$

      Generate $\bm z \sim p(\cdot \mid \bm\theta_i^{(1)})$
   }
   Set $w_i^{(1)} = 1/N$.
  }
  Set $\Sigma_1$ as twice the empirical variance of the
  $\bm \theta_i^{(1)}$'s

  \For{t=2 to T}{

    \For{i=1 to N}{

      \Repeat{$\rho[\eta(\bm y) , \eta (\bm z)] \leq \epsilon$}{
        Sample $\bm\theta_i^{*}$ from
        $\bm\theta_j^{(t-1)}$'s with probabilities $w_j^{(t-1)}$

        Generate
        $\bm\theta_i^{(t)} \sim N(\bm\theta_i^{*},\Sigma_{(t-1)})$ and
        $\bm z \sim p(\cdot \mid \bm\theta_i^{(t)})$
     }

     Set $w_i^{(t)} \propto \frac{\pi(\bm\theta_i^{(t)})}
     {\sum^N_{j=1}w_j^{(t-1)}\phi\{ (\Sigma_{t-1})^{-1/2} 
     (\bm\theta_i^{(t)} - \bm\theta_j^{(t-1)})\} }$.
    }
  Set $\Sigma_t$ as twice the weighted variance of the
  $\bm \theta_i^{(t)}$'s
  }

   \caption{ABC-PMC}
  \end{algorithm}
\newpage
\hfill\break
    \begin{figure}[H]
        \centering
        \includegraphics[width=9cm]{images/ABC-PMC.png}
        \caption{Comparison of ABC-PMC samples with the true posterior
        marginal distribution for $\theta_1$ (\textit{left}) and
        $\theta_2$ (\textit{right}) using $\epsilon = 0.1\%$.
        }
    \end{figure}

\section{Post-processing of ABC}

Instead of altering the ABC algorithm, methods have been
developed that improve the estimated posterior
by post-processing results of the ABC sampler. One
of this methods is the
\textit{local linear regression} proposed by
\citet{Beaumont2012}.
The idea is to use a weighted least squares regression
of $\bm \theta$ on $(\eta(\bm y) - \eta(\bm z))$, with weights
according to a chosen kernel.
\begin{equation}
\bm \theta^* = \bm \theta - (\eta(\bm y) - \eta(\bm z))^T\hat\beta
\end{equation}
Hence, this method adjusts the values of the sampled $\bm \theta$'s
by projecting them in the axis where the error is equal to
zero. Thus, the threshold of acceptance can be lessened without
harming the posterior approximation.
\hfill\break

    \begin{figure}[H]
        \centering
        \includegraphics[width=9cm]{images/RegressionABC1.png}
        \caption{Scatter plots of simulated $\theta_1$ and
        $(\eta(\bm y)-\eta(\bm z))$ for autocovariance with
        $lag = 1$. On the \textit{left} there are all the
        $N=10^6$ simulations, while on the \textit{right} only the
        accepted samples for $\epsilon = 10\%$ with the regression
        line.
        }
        \label{fig:regression}
    \end{figure}

Note that the linear regression is not done for every simulated
value, but only those that have
$\eta(\bm y) - \eta(\bm z) \leq \epsilon$. As
shown in 
Figure ~\ref{fig:regression}, after a certain error,
the scatter plot starts to present a non-linear behavior, which
is why the correction by \citet{Beaumont2012} is only local.
The results of this correction applied to the MA(2) model
are shown below, and they are indeed as good, if not better,
then when we used $\epsilon = 0.1\%$ in the ``vanilla'' algorithm.
\hfill\break
    \begin{figure}[H]
        \centering
        \includegraphics[width=9cm]{images/ABCRegression_Marginal.png}
        \caption{Comparison of ABC samples corrected through
        local linear regression versus the true marginal posterior
        distribution for $\theta_1$ (\textit{left}) and
        $\theta_2$ (\textit{right}) using $\epsilon = 10\%$.
        }
    \end{figure}

\citet{Blum2010}
proposed a nonlinear model with heteroskedasticity,
instead of the linear regression. In this approach,
the parameters are estimated by a neural network with
one hidden layer. This model reduces event further
the necessity of using low thresholds, thus accepting
more simulated values.

\section{Model Choice}
Model choice is part of the framework of Bayesian analysis.
In it, different models are compared and the end goal is
to evaluate the probability that a specific model generated the data.
Therefore, in addition to the parameters
of each model, the inference also estimates
a parameter
$\mathcal{M}$ that corresponds to the index of each specific
model, hence, $\mathcal{M}$ takes values in $\{1,2,...,m\}$, where
$m$ is the total number of models being compared.

The estimation of the posterior using ABC is straightfoward, the adpated algorithm is given below. Note that
$\bm \theta_m$ corresponds to the parameters of each $m$, and
$\eta(\bm z) = (\eta_1(\bm z),...,\eta_m(\bm z))$ is the concatenation
of the summary statistics used for each model.

\newpage
  \begin{algorithm}[H]
  \SetAlgoLined
  \For{i=1 to N}{
    \Repeat{$\rho[\eta(\bm y) , \eta (\bm z)] \leq \epsilon$}{
      Sample $m \sim \pi(\mathcal{M}=m)$

      Sample $\bm\theta_m \sim \pi_m(\bm \theta_m)$

      Generate $\bm z \sim p_m(\cdot \mid \bm\theta_m)$
   }
   Set $m^{(i)}=m$ and $\bm\theta^{(i)} = \bm\theta_m$
    
  }
   \caption{ABC Model Choice}
  \end{algorithm}
\hfill\break

  The posterior probability $\pi(\mathcal{M}=m \mid \bm y)$
  is the acceptance frequency from model m, given by
  \begin{equation}
  \frac{1}{N}\sum^N_{i=1}\mathbb I_{m^{(i)}=m}
  \end{equation}

We apply the ABC algorithm for the MA model, comparing
MA(1) and MA(2) as possible generators for the data. A common
metric in model choice is the Bayes Factor (BF), which in our
case is given by
\begin{equation}
BF = \frac{P(M_2 \mid y)}{P(M_1 \mid y)}  = 
\frac{\int \int P(y \mid \theta_1, \theta_2, M_2)\pi(\theta_1,\theta_2 \mid M_2)d\theta_1d\theta_2}
{\int P(y \mid \theta_1, M_1)\pi(\theta_1 \mid M_1)d\theta_1}
\end{equation}
where $M_1$ is MA(1) and $M_2$ is MA(2).
Two cases were simulated. First, a sample of length 100 was
generated using MA(2) with $(\theta_1,\theta_2) = (0.6,0.38)$. And
secondly, a sample of length 100 using MA(1) with $\theta_1 = 0.6$.
For each case, the real Bayes Factor was numerically obtained and 
compared to the one obtained through ABC, as shown below.

\hfill\break
    \begin{figure}[H]
        \centering
        \includegraphics[width=12cm]{images/ModelChoice_MA2.png}
        \caption{Barplots of evolution of Bayes factor approximations
        in terms of visits to models MA(1) and MA(2) using
        ABC using thresholds of 10, 1, 0.1 , 0.01 \% on autocovariance
        distance. The true model is MA(2) and the true Bayes factor is 0.952.
        }
    \end{figure}

    \begin{figure}[H]
        \centering
        \includegraphics[width=12cm]{images/ModelChoice_MA1.png}
        \caption{Barplots of evolution of Bayes factor approximations
        in terms of visits to models MA(1) and MA(2) using
        ABC using thresholds of 10, 1, 0.1 , 0.01 \% on autocovariance
        distance. The true model is MA(1) and the true Bayes factor is 0.943.
        }
    \end{figure}

The ABC algorithm is able to get an approximation of the real
Bayes factor, but even with lower thresholds it seems to plateau,
never actually reaching the real value, similar to what happened
to the posterior distribution of $\theta_2$.


%
% ---- Bibliography ----
%
% BibTeX users should specify biblikography style 'splncs04'.
% References will then be sorted and formatted in the correct style.
%
\bibliographystyle{apa}
\bibliography{abc}
%
\end{document}
