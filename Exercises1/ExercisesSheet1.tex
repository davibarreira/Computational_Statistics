\documentclass[12pt,letterpaper]{article}
\usepackage{./preamble}

%%%%%%%%%%%%%%%%%%%%%%%%%%%%%%%%%%%%%%%%%%
%%%% Edit These for yourself
%%%%%%%%%%%%%%%%%%%%%%%%%%%%%%%%%%%%%%%%%%
\newcommand\course{Computational Statistics}
\newcommand\hwnumber{1}
\newcommand\userID{Davi Sales Barreira}

\begin{document}
% \textbf{\Large Worksheet completed with Octave.}
Notation: here is a brief summary of the notation used in this worksheet.
\begin{itemize}
\item $ p(X=x)$ is equal to the probability density function;
\item Capital letters such as $X$ stand for the random variable.
\end{itemize}

\section*{Exercise 1 (Inversion and Rejection)}
\begin{enumerate}[leftmargin=!,labelindent=5pt]
    \item Let $ F_X(x) = \mathbb{P}(X \leq x) $ and $U \sim Unif[0,1]$:
        % \begin{flalign}
            $$ F_X(x) = 1 - e^{-\lambda(X-a)}\mathbb{I}_{\{X \geq a\}}=U $$
            $$ -\ln(1-U) = \lambda(x-a) $$
            $$ F_X^{-1}(U) = a - \frac{-\ln(1-U)}{\lambda} $$
    To simulate X from U, just simulate value from U and substitute in the
    formula above.
        % \end{flalign}
        
    \item
    $$p(X=x \mid a \leq X \leq b) = p(X=x, a \leq X \leq b)$$
\end{enumerate}

% \newpage
% \section*{Question 2}
% \setcounter{equation}{0}
% \begin{enumerate}[leftmargin=!,labelindent=5pt]
%     \item Equations from parts 1 and 2
%         \begin{enumerate}
%             \item Write the equation of the surface in the form $z = f(x, y)$.
%             \begin{flalign}
%                 f(x)=(x+a)(x+b)\\
%                 L' = {L}{\sqrt{1-\frac{v^2}{c^2}}} \\
%                 \lim_{x\to 0}{\frac{e^x-1}{2x}}\\
%                 \overset{\left[\frac{0}{0}\right]}{\underset{\mathrm{H}}{=}}\\
%                 \lim_{x\to 0}{\frac{e^x}{2}}={\frac{1}{2}}
%             \end{flalign}
        
%         \item Make inline math with dollar dollar y'all $woo$! Also centered equations. Tell LaTeX where you want to align equations with \&.
%             \begin{align*}
%                 f(x) &= x^2 \\
%                 g(x) &= \frac{1}{x} \\
%                 F(x) &= \int^a_b \frac{1}{3}x^3
%             \end{align*}
%         \end{enumerate}
%     \newpage
    
%     % \item Plots for Part 3
%     %     \begin{enumerate}
%     %         \item Good times.
%     %             \begin{figure}[H]
%     %                 \centering
%     %                 \includegraphics[width=15cm]{images/figure3.jpg}
%     %                 \caption{A meme.}
%     %                 \label{fig:3}
%     %             \end{figure}
%     %     \end{enumerate}
% \end{enumerate}
\end{document}